\documentclass[12pt]{article}
\usepackage{amsmath, amssymb}
\usepackage{geometry}
\geometry{margin=2.5cm}

\title{Evaluación Continua N°1}
\author{---}
\date{}

\begin{document}
\maketitle

\section*{Ejercicio 1}

Para un conjunto de datos cuantitativo 
\[
(x_1, x_2, \ldots, x_n),
\]
identifique las afirmaciones correctas:

\begin{enumerate}
    \item[a)] 
    \[
    \sum_{i=1}^{n-1}(x_i - \bar{x})^2 
    = \sum_{i=1}^{n-1} x_i^2 - n\bar{x}^2.
    \]

    \item[b)]
    \[
    \sum_{i=1}^{n}(x_i - \bar{x})^2 = 0.
    \]

    \item[c)]
    Si $(x_1, \ldots, x_n)$ es un conjunto de datos con media $\bar{x}$, entonces  
    \[
    (x_1+3, \ldots, x_n+3) 
    \quad\text{tiene media}\quad (\bar{x}-3).
    \]

    \item[d)]
    Todo conjunto de datos derivado de una variable cuantitativa tiene una media.
\end{enumerate}

\subsection*{Resolución y demostraciones}

\textbf{Afirmación (a): Falsa.}

La identidad correcta es:
\[
\sum_{i=1}^{n}(x_i - \bar{x})^2 = 
\sum_{i=1}^{n} x_i^2 - n\bar{x}^2.
\]

Demostración:
\[
\begin{aligned}
\sum_{i=1}^n (x_i - \bar{x})^2 
&= \sum_{i=1}^n \left( x_i^2 - 2x_i\bar{x} + \bar{x}^2 \right) \\
&= \sum_{i=1}^n x_i^2 - 2\bar{x}\sum_{i=1}^n x_i + n\bar{x}^2 \\
&= \sum_{i=1}^n x_i^2 - 2\bar{x}(n\bar{x}) + n\bar{x}^2 \\
&= \sum_{i=1}^n x_i^2 - n\bar{x}^2.
\end{aligned}
\]
Como el límite superior es $n$ y no $n-1$, la expresión dada es incorrecta.

\vspace{0.3cm}

\textbf{Afirmación (b): Falsa.}

La suma de cuadrados respecto a la media es cero únicamente si
\[
x_1 = x_2 = \cdots = x_n = \bar{x}.
\]
En un conjunto genérico esto no ocurre. Por tanto, la afirmación es falsa.

\vspace{0.3cm}

\textbf{Afirmación (c): Falsa.}

Sea el nuevo conjunto:
\[
y_i = x_i + 3.
\]
Entonces:
\[
\bar{y} = \frac{1}{n} \sum_{i=1}^n (x_i+3)
        = \bar{x} + 3.
\]
Es decir, sumar una constante aumenta la media en esa constante. No se resta.

\vspace{0.3cm}

\textbf{Afirmación (d): Verdadera.}

Todo conjunto cuantitativo con $n \ge 1$ posee media definida:
\[
\bar{x} = \frac{1}{n}\sum_{i=1}^n x_i.
\]

---

\section*{Ejercicio 2}

Dado un conjunto de datos y usando la variable \texttt{sc1}, se requiere calcular:

\begin{itemize}
    \item Promedio
    \item Desviación estándar
    \item Índice de asimetría (Skewness)
    \item Índice de curtosis (Kurtosis)
\end{itemize}

Los valores obtenidos mediante SPSS fueron:

\[
\text{Media} = 55.37, 
\quad
\text{Desviación estándar} = 23.03,
\]
\[
\text{Asimetría} = -0.19, 
\quad
\text{Curtosis} = -0.93.
\]

\subsection*{Interpretación}

\begin{itemize}
    \item La media indica un nivel promedio de 55.37 unidades en la variable.
    \item La desviación estándar muestra una dispersión considerable alrededor de la media.
    \item La asimetría negativa leve ($-0.19$) indica casi simetría con ligera inclinación a la izquierda.
    \item La curtosis negativa ($-0.93$) revela una distribución platicúrtica (más aplanada que la normal).
\end{itemize}

---

\end{document}



